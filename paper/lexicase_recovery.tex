% This is "sig-alternate.tex" V2.1 April 2013
% This file should be compiled with V2.5 of "sig-alternate.cls" May 2012
%
% This example file demonstrates the use of the 'sig-alternate.cls'
% V2.5 LaTeX2e document class file. It is for those submitting
% articles to ACM Conference Proceedings WHO DO NOT WISH TO
% STRICTLY ADHERE TO THE SIGS (PUBS-BOARD-ENDORSED) STYLE.
% The 'sig-alternate.cls' file will produce a similar-looking,
% albeit, 'tighter' paper resulting in, invariably, fewer pages.
%
%--------------------------
% This .tex file (and associated .cls V2.5) produces:
%       1) The Permission Statement
%       2) The Conference (location) Info information
%       3) The Copyright Line with ACM data
%       4) NO page numbers
%
% as against the acm_proc_article-sp.cls file which
% DOES NOT produce 1) thru' 3) above.
%
% Using 'sig-alternate.cls' you have control, however, from within
% the source .tex file, over both the CopyrightYear
% (defaulted to 200X) and the ACM Copyright Data
% (defaulted to X-XXXXX-XX-X/XX/XX).
% e.g.
% \CopyrightYear{2007} will cause 2007 to appear in the copyright line.
% \crdata{0-12345-67-8/90/12} will cause 0-12345-67-8/90/12 to appear in the copyright line.
%
% ---------------------------------------------------------------------------
% This .tex source is an example which *does* use
% the .bib file (from which the .bbl file % is produced).
% REMEMBER HOWEVER: After having produced the .bbl file,
% and prior to final submission, you *NEED* to 'insert'
% your .bbl file into your source .tex file so as to provide
% ONE 'self-contained' source file.
%
% ================= IF YOU HAVE QUESTIONS =======================
% Questions regarding the SIGS styles, SIGS policies and
% procedures, Conferences etc. should be sent to
% Adrienne Griscti (griscti@acm.org)
%
% Technical questions _only_ to
% Gerald Murray (murray@hq.acm.org)
% ===============================================================
%
% For tracking purposes - this is V2.0 - May 2012

\documentclass{sig-alternate-05-2015}

%\usepackage{tabu}
%\usepackage{multirow}

\usepackage{booktabs}


\begin{document}

% Copyright
\setcopyright{acmcopyright}
%\setcopyright{acmlicensed}
%\setcopyright{rightsretained}
%\setcopyright{usgov}
%\setcopyright{usgovmixed}
%\setcopyright{cagov}
%\setcopyright{cagovmixed}

\clubpenalty=10000
\widowpenalty = 10000


%Conference
\conferenceinfo{GECCO'16,} {July 20-24, 2016, Denver, Colorado, USA.}

% \acmPrice{\$15.00}

%
% --- Author Metadata here ---
\CopyrightYear{2016} % Allows default copyright year (20XX) to be over-ridden - IF NEED BE.
\crdata{TBA}  % Allows default copyright data (0-89791-88-6/97/05) to be over-ridden - IF NEED BE.
% --- End of Author Metadata ---

\title{Parent Selection and Diversification in Genetic Programming}

%
% You need the command \numberofauthors to handle the 'placement
% and alignment' of the authors beneath the title.
%
% For aesthetic reasons, we recommend 'three authors at a time'
% i.e. three 'name/affiliation blocks' be placed beneath the title.
%
% NOTE: You are NOT restricted in how many 'rows' of
% "name/affiliations" may appear. We just ask that you restrict
% the number of 'columns' to three.
%
% Because of the available 'opening page real-estate'
% we ask you to refrain from putting more than six authors
% (two rows with three columns) beneath the article title.
% More than six makes the first-page appear very cluttered indeed.
%
% Use the \alignauthor commands to handle the names
% and affiliations for an 'aesthetic maximum' of six authors.
% Add names, affiliations, addresses for
% the seventh etc. author(s) as the argument for the
% \additionalauthors command.
% These 'additional authors' will be output/set for you
% without further effort on your part as the last section in
% the body of your article BEFORE References or any Appendices.

\numberofauthors{3} %  in this sample file, there are a *total*
% of EIGHT authors. SIX appear on the 'first-page' (for formatting
% reasons) and the remaining two appear in the \additionalauthors section.
%
\author{
% 1st. author
\alignauthor
Author Omitted\\
       \affaddr{.}\\
       \affaddr{.}\\
       \affaddr{.}\\
       \email{.}
% 2nd. author
\alignauthor
Author Omitted\\
       \affaddr{.}\\
       \affaddr{.}\\
       \affaddr{.}\\
       \email{.}
% 3rd. author
\alignauthor
Author Omitted\\
       \affaddr{.}\\
       \affaddr{.}\\
       \affaddr{.}\\
       \email{.}
}

%\alignauthor
%Thomas Helmuth\\
%       \affaddr{Computer Science}\\
%       \affaddr{Washington and Lee Univ}\\
%       \affaddr{Lexington, Virginia}\\
%       \email{helmutht@wlu.edu}
%% 2nd. author
%\alignauthor
%Nicholas Freitag McPhee\\
%       \affaddr{Div of Sci and Math}\\
%       \affaddr{Univ of MN, Morris}\\
%       \affaddr{Morris, MN 56267}\\
%       \email{mcphee@morris.umn.edu}
%% 3rd. author
%\alignauthor Lee Spector\\
%       \affaddr{Cognitive Science}\\
%       \affaddr{Hampshire College}\\
%       \affaddr{Amherst, MA}\\
%       \email{lspector@hampshire.edu}
%}

\maketitle
\begin{abstract}
More things!
\end{abstract}

%
%  Use this command to print the description
%
\printccsdesc

% We no longer use \terms command
%\terms{Theory}

\keywords{lexicase selection, hyperselection, PushGP, other stuff}

\section{Introduction}
\label{sec:introduction}

I bet we start here!

Lexicase selection \cite{Spector:2012:GECCOcompANEW} is nifty, eh?

Hyperselection provided initial motivation~\cite{Helmuth:2016:GECCO}, but later we became interested more generally in what lexicase and tournament do differently starting with the same population.

[TMH: I'm not sure if we want to talk about error diversity in the Intro or somewhere else. But, the following paragraph could be moved to be the first paragraph of Experimental Design if we don't need to talk about diversity earlier]

In this paper we concentrate on diversity measures related to the outputs of the programs. One such diversity measure, \textit{behavioral diversity}, has been shown to have correlation with problem-solving performance \cite{Jackson:2010:PPSN}. In behavioral diversity, the output of each program is recorded on each training case input and stored as a \textit{behavior vector}. Behavioral diversity is then the percentage of distinct behavior vectors in the population. \textit{Error diversity}, a slight variation of behavioral diversity, considers the percentage of distinct error vectors in the population where each error vector is computed by applying the error function to each output in the behavior vector. We believe error diversity does a good job of measuring how well evolution is exploring meaningful differences between programs that might be lost with a diversity measure that only takes into account syntactic (genotypic) diversity of the population, where two wildly different programs may actually compute the same function.

\section{Experimental Design}

Previous work has shown that using lexicase selection results in higher population error diversity than tournament selection across a variety of problems \cite{Helmuth:2015:GPTP, Helmuth:2015:ieeeTEC}. These papers examined the diversity of entire GP runs, each starting with a different initial population and random number seed.

Here we examine the effcts of these parent selection methods on population error diveristy starting from specific population conditions besides a random initial population. In particular, we want to see how each method changes diversity in populations that occur naturally during an evolutionary run.

In order to produce the populations on which to experiment, we started GP runs and let them continue until they met certain stopping conditions; we then stored those populations and later conducted multiple trials with different random number seeds starting with those stored populations. We used three different stopping conditions in order to generate naturally occuring populations with interesting properties:
\begin{enumerate}
\item
In a run using lexicase selection, we stopped if the population error diversity was greater than 0.9. This results in very diverse populations, allowing us to observe whether evolution is able to maintain such high diversity in the following generations.

\item
In a run using tournament selection, we stopped if the population error diversity was less than 0.15. These populations allow us to see if methods promote diversification starting from such undiverse populations. They also allow us to see if methods perform differently on a population produced by tournament selection versus one produced by lexicase selection.

\item
As described above, we were initially motivated here by observations of runs using lexicase selection that underwent major drops in diversity following hyperselection events, where one or a few individuals in the population received the majority of the parent selections in a generation. We had anecdotally noticed rapid diversity recovery following these events, but not examined them systematically~\cite{Helmuth:2016:GECCO}.

In this condition, we stopped a run using lexicase selection when the error diversity reached a level at least 0.25 less than it had been at some point in the previous 10 generations. This allowed us to detect populations that had recently undergone large drops in diversity. We do not definitively know whether those drops are related to hyperselection events, but we expect that they are.

\end{enumerate}
In all three conditions, we only considered populations occuring after the first 10 generations in order to give evolution a chance to settle down after the extreme shifts that can happen at the beginning of a run.

In each trial, we continued running GP on a stored population for 20 generations and recorded the population error diversity. For each parent selection setting (lexicase and tournament selections), we conducted 100 trials with different random number seeds from each stored population.

We conducted these tests on two problems taken from a recent program synthesis benchmark suite \cite{Helmuth:2015:GECCO}. The first problem, Replace Space With Newline (RSWN), searches for a program that takes as input a string and both prints the string after replacing all of the spaces in the input with newline characters and functionally returns the number of non-whitespace characters in the string. Previous examinations of error vector diversity on the RSWN problem indicate that lexicase selection maintains significantly higher diversity than tournament selection, which across 100 runs never achieved a median diversity higher than 0.25 \cite{Helmuth:2015:GPTP}.

The second problem, Double Letters, asks for a program that takes a string as input and prints the string after doubling every alphabetic character and tripling every exclamation point. All other characters should be printed once. As with the RSWN problem, lexicase selection consistently achieves high diversity on this problem. Differently than RSWN, runs using tournament selection show slow but steady increases in diversity, though not approaching that of lexicase selection runs \cite{Helmuth:2015:GPTP}.

\begin{table}[t]
\centering
\caption{PushGP parameters}
\label{table:parameters}
%\rowcolors{3}{Gray}{white}
\begin{tabular}{l r}
\toprule
\textbf{Parameter} & \textbf{Value} \tabularnewline
\midrule
runs per problem/parent selection combination & 100 \tabularnewline
population size & 1000 \tabularnewline
maximum genome size & 1600 \tabularnewline
maximum initial genome size & 400 \tabularnewline
%alternation rate & 0.01 \tabularnewline
%alignment deviation & 10 \tabularnewline
%uniform mutation rate & 0.01 \tabularnewline
%uniform close mutation rate & 0.1 \tabularnewline
\midrule
\textbf{Genetic Operator} & \textbf{Prob} \tabularnewline
\midrule
alternation & 0.2 \tabularnewline
uniform mutation & 0.2 \tabularnewline
uniform close mutation & 0.1 \tabularnewline
alternation followed by uniform mutation & 0.5 \tabularnewline
\bottomrule
\end{tabular}
\end{table}

For our experiments we used PushGP \cite{spector:2002:GPEM, 1068292}, a stack-based genetic programming system.\footnote{Lexicase selection has also been shown to be effective in tree-based genetic programming \cite{Helmuth:2015:ieeeTEC, Krawiec:2015:GECCO:smgpWorkshop}.} PushGP supports a variety of control structures and multiple data types, making it a good choice for program synthesis tasks such as the problems we explore here.
Except for parent selection, we used the exact same PushGP parameters in both the initial runs used to store interesting populations as well as the continuations of the stored populations. We give the most relevant parameters in Table~\ref{table:parameters}. The parameters not listed here exactly follow those used in the experiments in  \cite{Helmuth:2015:dissertation}.

These runs use the most recent version of PushGP, in which individuals are stored as linear genomes that we translate into hierarchical Push programs prior to execution \cite{Helmuth:2015:dissertation}. These linear genomes admit a range of uniform genetic operators; we use four, listed in Table~\ref{table:parameters} with their probabilities. Alternation is a linear crossover operator modeled after the sexual portion of ULTRA \cite{Spector:2013:GPTP}. Uniform mutation may replace each instruction with 1\% probability. Uniform close mutation may add or remove parentheses from the program. Finally, the last operator runs alternation on two parents and then uniform mutation on that child to produce a new child.


\section{Results}
\label{sec:results}

Using the techniques presented in the previous section we obtained populations on which to perform continuation experiments. For each combination of the 2 problems and 3 stopping conditions we stored populations from 2 runs, for a total of 12 populations. In the following subsections we group the results based on the stopping conditions, since they produce the most similar populations.

Starting with each stored population we conducted 100 ``continuation'' GP runs with lexicase selection and 100 with tournament selection. We let each continuation go for 20 generations, and plot the population error diversity across the runs. In particular, each figure has a standard box-and-whisker plot for each generation, with the box showing the median and quartiles. The whiskers stretch to the maximum and minimum values, ignoring outliers. In each figure we also plot the error diversity of each individual run at each generation, giving another way of visualizing the spread of diversities across runs and making it easier to trace runs with outliers.

Note that in a few settings, one or two runs out of 100 found solutions to the problem before the end of 20 generations. In these cases, we terminate the runs, and they do not contribute data past their termination generation. We do not believe these solutions have a large effect on the plots since no plot had more than two of these early-terminating runs.



\subsection{Starting with high diversity}
\label{sec:highDiversityResults}

In this subsection we continue runs using stored populations evolved using lexicase selection that achieved error diversity greater than 0.9. As such, the initial populations of the runs have very high diversity, with most individuals producing distinct error vectors.

\begin{figure*}
	\includegraphics{../figures/RSWN_high_diversity}
	\vspace{-1 cm}
	\caption{Error diversity over 100 continuations of the RSWN problem with both lexicase and tournament selections, starting from populations with high diversity naturally occuring in a run using lexicase selection.}
	\label{fig:RSWNhighDiversity}
\end{figure*}

\begin{figure*}
	\includegraphics{../figures/DL_high_diversity}
	\vspace{-1 cm}
	\caption{Error diversity over 100 continuations of the double-letters problem with both lexicase and tournament selections, starting from populations with high diversity naturally occuring in a run using lexicase selection.}
	\label{fig:DLhighDiversity}
\end{figure*}

Figure~\ref{fig:RSWNhighDiversity} plots the continued runs started from two populations (C and D) stored from GP evolving on the RSWN problem. Lexicase selection consistently maintains high levels of diversity starting from both populations, with little variance. On the other hand, both plots show runs using tournament selection quickly losing significant diversity within the first 5 to 10 generations of the continuation, dropping from over 90\% distinct error vectors down to around 50\% distinct error vectors. Interestingly, the tournament selection runs on Population C show large differences in diversity the last 10 generations, with some becoming even less diverse while others recovering much of the lost diversity. On the other hand, the tournament selection runs on Population D maintain much more consistent diversity, with most runs having between 40\% and 60\% diversity in the remaining generations. It is unclear to us why these two populations result in such different diversity plots for tournament selection, but we assume it has to do with the composition of the stored population.

Figure~\ref{fig:DLhighDiversity}, which plots the diversities of continuations of two populations (I and J) on the Double Letters problem, shows similar trends in both lexicase selection and tournament selection. Note that tournament selection maintains higher diversity on this problem than on the RSWN problem, though not as high as lexicase selection. This trend mirrors what has been previously seen on full GP runs \cite{Helmuth:2015:GPTP}.

\textit{[should the next two paragraphs be in Discussion?]}
The continuations starting from high-diversity populations clearly show that lexicase selection can maintain a high population diversity while tournament selection cannot reliably do so.

One interesting observation in these plots is the occasional steep drop in diversity in a small number of runs using lexicase selection, which can be seen especially clearly in populations D and I. Based on similar runs we have encountered previously, we would guess that these runs underwent hyperselection events in which one or a small number of individuals were selected to make most of the children in a single generation. Hyperselection events can, understandably, lead to diversity crashes since most of the individuals in the population are closely related. Interestingly, previous work has shown that such events are neither a driving force or a hinderance in runs using lexicase selection \cite{Helmuth:2016:GECCO}.


\subsection{Starting with low diversity}
\label{sec:lowDiversityResults}

Next, we present continuations of runs that start from populations exhibiting very low population diversity, at most 0.15. In other words, most of the individuals in these populations produced the same error vectors. These populations were stored from runs using tournament selection, since we were not able to achieve population diversity this low in a run using lexicase selection. Additionally, this will allow us to examine whether the parent selection technique used to produce the initial populations has effect on the continued diversity.

\begin{figure*}
	\includegraphics{../figures/RSWN_low_diversity}
	\vspace{-1 cm}
	\caption{Error diversity over 100 continuations of the RSWN problem with both lexicase and tournament selections, starting from populations with low diversity  naturally occuring in a run using tournament selection.}
	\label{fig:RSWNlowDiversity}
\end{figure*}

\begin{figure*}
	\includegraphics{../figures/DL_low_diversity}
	\vspace{-1 cm}
	\caption{Error diversity over 100 continuations of the double-letters problem with both lexicase and tournament selections, starting from populations with low diversity naturally occuring in a run using tournament selection.}
	\label{fig:DLlowDiversity}
\end{figure*}

Figure~\ref{fig:RSWNlowDiversity} plots diversity of runs starting from populations A and B on the RSWN problem. Starting from both populations, tournament selection does not increase diversity across the 20 generations except for a handful of outlier runs. The continuations using lexicase selection increase the median diversity across runs rapidly, with over 50\% unique error vectors by generation 8 using population A and generation 4 using population B. For population A, lexicase selection runs continue to steadily rise in diversity over the 20 generations. On the other hand, many runs starting with population B undergo steep drops in diversity, such that by generation 9 the lower quartile of diversity falls precipitously from around 60\% to around 35\%. The indivdiually plotted run diversities show that many runs continue to see single-generation drops in diversity throughout the 20 generations. We believe this population was likely contained one or more individuals that, when varied in the right way, produce a child that dominates the rest of the population, leading to many hyperselection events and therefore drops in diversity. Even with these drops in diversity, lexicase selection maintains higher diversity than tournament selection in the majority of continuations.

We present continuations of low-diversity populations (G and H) evolved on the Double Letters problem in Figure~\ref{fig:DLlowDiversity}.



\subsection{Starting after a diversity crash}
\label{sec:crashDiversityResults}

\begin{figure*}
	\includegraphics{../figures/RSWN_diversity_crash}
	\vspace{-1 cm}
	\caption{Error diversity over 100 continuations of the RSWN problem with both lexicase and tournament selections, starting from populations that had lost diversity in a diversity crash in a lexicase selection run.}
	\label{fig:RSWNdiversityCrash}
\end{figure*}

\begin{figure*}
	\includegraphics{../figures/DL_diversity_crash}
	\vspace{-1 cm}
	\caption{Error diversity over 100 continuations of the double-letters problem with both lexicase and tournament selections, starting from populations that had lost diversity in a diversity crash in a lexicase selection run.}
	\label{fig:DLdiversityCrash}
\end{figure*}


\section{Discussion}

Why does lexicase do so much better at diversification??? Does it have to do with specialists? Probably! Does it have to do with concentrating on individual test cases or combinations thereof? Probabily!

Nic:

    Some of the individual drops in lexicase diversity are likely due to hyperselection events. Some of those then recovered, but it's not clear that all did; not sure what we say about that part.

    The details clearly depend on the problem and the run (i.e., the starting point). Not surprising, but definitely visible in the data.



\section{Conclusions}
\label{sec:conclusions}

I'm hoping we have conclusions.

\section*{Acknowledgments}
Lots of cool people helped us.

% The following two commands are all you need in the
% initial runs of your .tex file to
% produce the bibliography for the citations in your paper.
\bibliographystyle{abbrv}
\bibliography{lexicase_recovery}  % sigproc.bib is the name of the Bibliography in this case
% You must have a proper ".bib" file
%  and remember to run:
% latex bibtex latex latex
% to resolve all references
%
% ACM needs 'a single self-contained file'!
%


% \balancecolumns % GM June 2007
\end{document}
